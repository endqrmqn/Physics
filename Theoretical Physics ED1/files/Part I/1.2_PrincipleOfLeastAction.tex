\documentclass[../../Main.tex]{subfiles}

\begin{document}
\chapter{The Principle of Least Action}

\intro{
    Nature, as we come to know it in the classical sense, is quite lazy. Fortunately for us, the laziness allows us to predict the behavior of systems without knowing every minute detail. The principle of least action states that nature chooses the path of least action, where action is defined as the integral of the Lagrangian over time. This chapter explores the principle of least action, its mathematical formulation, and its implications in classical mechanics.
}

\section{Calculus of Variations}
% TODO: Write content for Calculus of Variations.

\section{Euler–Lagrange Equations}
% TODO: Write content for Euler–Lagrange Equations.

\section{Constraints and Lagrange Multipliers}
% TODO: Write content for Constraints and Lagrange Multipliers.

\section{D’Alembert’s Principle}
% TODO: Write content for D’Alembert’s Principle.

\section{Energy, Momentum, and Cyclic Coordinates}
% TODO: Write content for Energy, Momentum, and Cyclic Coordinates.

\section{Summary}
% Summarize key ideas.
\end{document}
