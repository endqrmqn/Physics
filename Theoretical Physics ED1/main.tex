\documentclass[A4,12pt,twoside]{book}
\usepackage{amd}

% %--------------------------------------------------------------------------
% %         General Setting
% %--------------------------------------------------------------------------

\graphicspath{{Images/}{../Images/}} %Path of figures
\setkeys{Gin}{width=0.85\textwidth} %Size of figures
\setlength{\cftbeforechapskip}{3pt} %space between items in toc
\setlength{\parindent}{0.5cm} % Idk
\input{theorems.tex} % Theorems styles and colors
\usepackage[english]{babel} %Language

\setlist[itemize]{itemsep=5pt} % Adjust the length as needed
\setlist[enumerate]{itemsep=5pt} % Adjust the length as needed



% \usepackage{lmodern} %  Latin Modern font
% \usepackage{newtxtext,newtxmath}




% %--------------------------------------------------------------------------
% %         General Informations
% %--------------------------------------------------------------------------
\newcommand{\BigTitle}{Theoretical Physics}

\newcommand{\LittleTitle}{
    For The Motivated Undergraduate
    }

    
\begin{document}

% %--------------------------------------------------------------------------
% %         First pages 
% %--------------------------------------------------------------------------
\newgeometry{top=8cm,bottom=.5in,left=2cm,right=2cm}
\subfile{files/0.0.0.titlepage}
\restoregeometry
\subfile{files/0.Preface}
\subfile{files/0.zommaire}

% %--------------------------------------------------------------------------
% %         Core of the document 
% %--------------------------------------------------------------------------
\part{Mechanics}

\subfile{files/Part I/1.1_StructureOfSpaceTime}
%   Chapter 1 — The Structure of Space and Time
%   - Coordinate systems and transformations
%   - Galilean invariance
%   - Inertial vs non-inertial frames
%   - Pseudo forces and rotating frames
%   - Introduction to generalized coordinates

\subfile{files/Part I/1.2_PrincipleOfLeastAction}
%   Chapter 2 — The Principle of Least Action
%   - Calculus of variations
%   - Euler–Lagrange equations
%   - Constraints and Lagrange multipliers
%   - D’Alembert’s principle
%   - Energy, momentum, and cyclic coordinates

\subfile{files/Part I/1.3_LagrangianApplications}
%   Chapter 3 — Applications of Lagrangian Mechanics
%   - Simple and coupled oscillators
%   - The double pendulum
%   - Atwood’s machine and generalized coordinates
%   - Small-angle approximations and linearization

\subfile{files/Part I/1.4_ConservationAndSymmetry}
%   Chapter 4 — Conservation Laws and Symmetry
%   - Noether’s theorem in classical mechanics
%   - Translation, rotation, and time-shift symmetries
%   - Energy–momentum tensor (mechanical form)
%   - Central potentials and effective potentials

\subfile{files/Part I/1.5_HamiltonianMechanics}
%   Chapter 5 — Hamiltonian Mechanics
%   - Legendre transforms
%   - Hamilton’s equations
%   - Poisson brackets and canonical structure
%   - Constants of motion and integrability
%   - Phase-space flow and Liouville’s theorem

\subfile{files/Part I/1.6_CanonicalTransformations}
%   Chapter 6 — Canonical Transformations and Hamilton–Jacobi Theory
%   - Generating functions (F₁–F₄ types)
%   - Canonical invariants
%   - The Hamilton–Jacobi equation
%   - Action–angle variables and integrable systems

\subfile{files/Part I/1.7_RigidBodyMotion}
%   Chapter 7 — Rigid-Body Motion
%   - Rotation matrices and Euler angles
%   - Angular velocity vector and moment of inertia tensor
%   - Euler’s equations for rigid-body dynamics
%   - Symmetric top and precession
%   - Gyroscopic effects and stability

\subfile{files/Part I/1.8_SmallOscillations}
%   Chapter 8 — Small Oscillations and Normal Modes
%   - Equilibrium and linearization
%   - Coupled harmonic oscillators
%   - Matrix formulation and eigenvalue problems
%   - Normal coordinates, orthogonality, and modal analysis

\subfile{files/Part I/1.9_CentralForceMotion}
%   Chapter 9 — Central Force Motion
%   - Motion in a plane and conservation of angular momentum
%   - Effective potential and orbital shapes
%   - Kepler problem and conic sections
%   - Scattering and Rutherford formula
%   - Perturbations and precession (e.g., Mercury)

\subfile{files/Part I/1.10_ContinuousSystems}
%   Chapter 10 — Continuous and Approximate Systems (Optional Bridge to Fields)
%   - Discrete-to-continuous limit (mass–spring chain → wave equation)
%   - Lagrangian density for a vibrating string
%   - Energy and momentum in continuous media
%   - Prelude to field theory

\subfile{files/Part I/1.11_MechanicsExercises}
%   Chapter 11 — Exercises for Part I: Mechanics
%   - 50 original problems without solutions
%   - Increasing difficulty across conceptual, computational, and theoretical levels
%   - Integrative challenges combining symmetry, dynamics, and geometry


\part{Fields and Waves}

\subfile{files/Part II/2.1_FieldConcept}
%   Chapter 1 — The Field Concept
%   - Motivation for fields and the continuum limit
%   - Scalar and vector fields
%   - Field energy and flux
%   - Local interactions and superposition

\subfile{files/Part II/2.2_WaveEquation}
%   Chapter 2 — The Wave Equation
%   - Derivation in one and three dimensions
%   - Boundary conditions
%   - Standing waves and superposition
%   - Energy transport and group velocity

\subfile{files/Part II/2.3_ElectromagneticFields}
%   Chapter 3 — Electromagnetic Fields
%   - Maxwell’s equations
%   - Potentials and gauge freedom
%   - Plane-wave solutions in vacuum
%   - Connection between electricity and magnetism

\subfile{files/Part II/2.4_FieldEnergyMomentum}
%   Chapter 4 — Energy, Momentum, and Stress in Fields
%   - Poynting vector and energy conservation
%   - Field momentum and stress tensor
%   - Relation to Noether’s theorem

\subfile{files/Part II/2.5_PolarizationAndMedia}
%   Chapter 5 — Polarization, Media, and Boundaries
%   - Dielectrics and conductors
%   - Boundary conditions
%   - Reflection, refraction, and polarization

\subfile{files/Part II/2.6_WavesAndRadiation}
%   Chapter 6 — Electromagnetic Waves and Radiation
%   - Plane and spherical waves
%   - Polarization states
%   - Radiation from accelerating charges
%   - Dipole fields and radiation patterns

\subfile{files/Part II/2.7_RelativisticElectrodynamics}
%   Chapter 7 — The Relativistic Structure of Electrodynamics
%   - Covariant form of Maxwell’s equations
%   - Four-vectors and the field tensor
%   - Lorentz invariance and transformations

\subfile{files/Part II/2.8_WavesInMedia}
%   Chapter 8 — Waves in Continuous Media
%   - Elastic and acoustic waves
%   - Dispersion, phase, and group velocity
%   - Plasma oscillations

\subfile{files/Part II/2.9_InterferenceDiffraction}
%   Chapter 9 — Interference and Diffraction
%   - Huygens–Fresnel principle
%   - Fraunhofer and Fresnel diffraction
%   - Coherence and interference patterns

\subfile{files/Part II/2.10_FieldTheoryFoundation}
%   Chapter 10 — Mathematical Foundations of Field Theory
%   - Lagrangian density for fields
%   - Euler–Lagrange equations in field form
%   - Gauge invariance and conserved currents

\subfile{files/Part II/2.11_FieldExercises}
%   Chapter 11 — Exercises for Part II: Fields and Waves
%   - 50 original problems without solutions
%   - Conceptual, computational, and theoretical variety
%   - Integrative challenges spanning waves and electromagnetism


\part{Fluids, Optics, and Heat}

\subfile{files/Part III/3.1_FluidStaticsKinematics}
%   Chapter 1 — Fluid Statics and Kinematics
%   - Density, pressure, and velocity fields
%   - Streamlines and vorticity
%   - Continuity equation and flow visualization

\subfile{files/Part III/3.2_EquationsOfFluidMotion}
%   Chapter 2 — Equations of Fluid Motion
%   - Euler and Navier–Stokes equations
%   - Stress tensor and viscosity
%   - Boundary layers and flow regimes

\subfile{files/Part III/3.3_CompressibleIncompressibleFlows}
%   Chapter 3 — Incompressible and Compressible Flows
%   - Bernoulli’s equation
%   - Sound waves in fluids
%   - Shock waves and Mach number

\subfile{files/Part III/3.4_ThermodynamicSystems}
%   Chapter 4 — Thermodynamic Systems and Equations of State
%   - First and second laws
%   - State variables and processes
%   - Reversible and irreversible transformations

\subfile{files/Part III/3.5_StatisticalThermodynamics}
%   Chapter 5 — Statistical Foundations of Thermodynamics
%   - Microstates and ensembles
%   - Boltzmann distribution and partition function
%   - Entropy and temperature from probability

\subfile{files/Part III/3.6_TransportPhenomena}
%   Chapter 6 — Transport Phenomena
%   - Diffusion, conduction, and viscosity
%   - Fick’s and Fourier’s laws
%   - Continuity and constitutive relations

\subfile{files/Part III/3.7_WaveOptics}
%   Chapter 7 — Wave Optics
%   - Interference and diffraction
%   - Polarization and coherence
%   - Optical path and intensity distributions

\subfile{files/Part III/3.8_GeometricalOptics}
%   Chapter 8 — Geometrical Optics
%   - Eikonal approximation and ray tracing
%   - Lenses, mirrors, and image formation
%   - Connection to Hamilton–Jacobi formalism

\subfile{files/Part III/3.9_ThermalRadiation}
%   Chapter 9 — Blackbody and Thermal Radiation
%   - Planck’s law and Stefan–Boltzmann relation
%   - Photon gas and spectral energy density
%   - Classical versus quantum limits

\subfile{files/Part III/3.10_NonEquilibriumProcesses}
%   Chapter 10 — Non-Equilibrium and Irreversibility
%   - Entropy production and Onsager relations
%   - Fluctuation–dissipation theorem
%   - Irreversible thermodynamics

\subfile{files/Part III/3.11_FluidOpticsHeatExercises}
%   Chapter 11 — Exercises for Part III: Fluids, Optics, and Heat
%   - 50 original problems without solutions
%   - Integrating thermodynamics, optics, and continuum physics


\part{Relativity}

\part{Quantum}


% %--------------------------------------------------------------------------
% %         Bibliographie 
% %--------------------------------------------------------------------------
\nocite{*} % to cite evey things, else cite each on using : \cite{ifrs17}. 
\printbibliography %to print bibliographie

\end{document}
